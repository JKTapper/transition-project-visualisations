\documentclass{article}
\usepackage{graphicx} % Required for inserting images
\usepackage{amssymb}
\usepackage{amsmath}
\usepackage{amsthm}
\newtheorem{theorem}{Theorem}[section]
\newtheorem{corollary}{Corollary}[theorem]
\newtheorem{lemma}[theorem]{Lemma}

\title{Finite Symmetric Zero-Sum Win Loss Draw Games}
\author{Joseph Tapper}
\date{November 2025}

\begin{document}

\maketitle

\section{Introduction}

\paragraph{This article explores a class of games defined by the following properties:}
\begin{enumerate}
    \item There are 2 players.
    \item Each player can choose from the same set of moves \(\{x_1,x_2,\dots,x_n\}\).
    \item Neither player knows what the other will play before making their choice.
    \item Either one player wins and the other loses, leading to payoffs of 1 and -1 respectively, or they draw and each receive a payoff of 0.
\end{enumerate}
\paragraph{Now we establish some notation:}
\paragraph{\(P\) is the payoff function. I.e if \(P(x_i,x_j) = p\) then that means that when player \(A\) plays \(x_i\) and player \(B\) plays \(x_j\) then player \(A\) receives payoff p. Note that, because our games are zero sum, \(P(x_i,x_j) = -P(x_j,x_i) \forall i,j \in (1,\dots,n)\). A game \(G\) is fully defined by its payoff function \(P_G\).}

\paragraph{\(S\) is the set of possible mixed strategies. A mixed strategy is a probability distribution over available moves. A player playing a mixed strategy will select their move randomly using the probability distribution. If \(G\) has moves \(\{x_1,x_2,\dots,x_n\}\) then \(S = \{(x_1,x_2,\dots,x_n) \in \mathbb{R}^n \mid \sum_{i=1}^nx_i = 1, x_i \in [0,1] \forall i\}\).}

\paragraph{We can now extend \(P\). We define \(P: S\times S\mapsto [0,1]\) such that \(P(s_1,s_2)\) is equal to the expected payoff for a player using strategy \(s_1\) against another player using \(s_2\). It is simple to prove that \(P(s_1,s_2) = -P(s_2,s_1)\).}

\paragraph{We call \(n \in S\) an equilibrium strategy if the scenario when both players play \(n\) is a Nash equilibrium. This means that neither player can gain an advantage by changing their strategy. Formally, \(n\) is an equilibrium strategy if \(P(n,n) \geq P(s,n) \forall s \in S\). The Equilibrium space \(N \subset S\) is the set of equilibrium strategies within S.}

\section{Basic properties of the equilibrium space}

\theorem{The equilibrium space \(N\) is non empty.}
\begin{proof}
    placeholder
\end{proof}
\end{document}
